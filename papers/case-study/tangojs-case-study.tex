% !TEX encoding = UTF-8 Unicode

\documentclass[10pt]{article}

\usepackage[pdftex]{graphicx}
\graphicspath{{./figures/}}

\usepackage[T1]{fontenc}
\usepackage[utf8]{inputenc}
\usepackage[english]{babel}
\usepackage{lmodern}

\usepackage{moreverb}
\usepackage{csagh}
\usepackage{nohyperref}
% \usepackage{natbib}
\usepackage{tikz}
\usepackage{textcomp}
% \usepackage{listings}
% \usepackage{tikz}
\usepackage{standalone}
\usepackage{mathrsfs}
\usepackage{amssymb}
\usepackage{empheq}
\usepackage{braket}
\usepackage{empheq}
\usepackage{float}
\usepackage{color}
\usepackage{listings}
\newcommand{\passthrough}[1]{#1}
\usepackage{gensymb}
\usepackage{caption}

\setkeys{Gin}{width=\maxwidth,height=\maxheight,keepaspectratio}

\def\tightlist{}

\begin{document}
\begin{opening}

\title{A case study of a web-based control panel built with TangoJS}

\author[liszcz.michal@gmail.com]{Michał Liszcz}

\author[AGH University of Science and Technology, ACC CYFRONET AGH, Kraków, Poland, funika@agh.edu.pl]{Włodzimierz Funika}

\begin{abstract}
    
Scientific and industrial hardware installations are becoming more and more
complex and require sophisticated methods of control. To address these needs,
Tango Controls system has been developed. As Tango is a CORBA-based software,
developers cannot build control applications using web technologies.
Recently the TangoJS project has been developed as an attempt to integrate
Tango with web browsers. This paper describes the TangoJS' design and
architecture, evaluates TangoJS against existing solutions,
presents a case study of building an interactive control-panel application
and discusses possible production-grade deployment scenarios for TangoJS
applications.

\end{abstract}

\keywords{SCADA, web, browser, Javascript, Tango, HTML, CORBA, REST}

\end{opening}

\tikzset{font=\Large}

\input{tangojs-case-study-sections.tex}

\begin{acknowledgements}
 The research presented in this paper was partially supported by \ldots
\end{acknowledgements}

% \nocite{*} %REMOVE

\bibliographystyle{cs-agh}
\bibliography{tangojs-case-study-bibliography}

\end{document}

